\documentclass[12pt,a4paper]{article}
\usepackage[utf8]{inputenc}
\usepackage{amsmath}
\usepackage{amsfonts}
\usepackage{amssymb}
\usepackage{graphicx}
\author{Ray Johns}
\title{C++ Legs}
\begin{document}
\title{Assignment 1}
\maketitle

\section{Q1}

The program searches 39999 numbers (1 through 40000, non-inclusive of the endpoint). \\
There are four perfect numbers in this interval, namely 6, 28, 496, and 8128.

\section{Q3}

It takes less work to evaluate the \texttt{isPerfect} algorithm on the number 10 than on the number 1000. That's because it calls \texttt{divisorSum}, which checks to see whether any number in the range $[1, n)$ is a divisor of $n$ for each $n$ between 1 and the stopping point. There is less computation required when the stopping point is 10 than when it is 1000.

It takes more work to search the range $[1000, 2000)$ than the range $[1, 1000)$, because of the computation required to evaluate \texttt{divisorSum}: since 1000 is greater than 1 by three orders of magnitude, there are correspondingly more divisors to check for each $n$ in the range.
\end{document}